\begin{rubric}{Notable Projects}
%     \entry*
%     \textbf{ ResNet over Convolutional Networks | Course Project - IIT Delhi (Prof. Parag Singla) (Feb, 2024 - March, 2024):}\par
%      Have implemented ResNet for Image Classification from scratch in PyTorch, studied the effect of different normalisation schemes, viz. Batch
% Normalization, Instance Normalization, Batch-Instance Normalization, Layer Normalization, and Group Normalization within ResNet.\par
%      Experimented with a dataset on Indian Birds species classification. Compared the impact of BN and GN, used Grad-CAM for producing visual-explanations
% from the ResNet model.

\entry*
\textbf{ResNet and Seq2Seq models (Prof. Parag Singla) (Feb, 2024 - April, 2024):}\par
Spearheaded the development of a ResNet model from scratch in PyTorch for Indian Birds species classification, integrating advanced normalization schemes and developed Sequence to Sequence translation models for text to Math solver sequence, included attention and experimented with teacher forcing for models with Pre-trained Bert-uncased encoder.\par
\textbf{GitHub Repository:} \href{https://github.com/nikhiltej5/Resnet_and_LSTMs.git}{\texttt{click here}}

    % \entry*
    % \textbf{Rollerball AI bot | Course Project - IIT Delhi (Prof. Mausam) (Sept, 2023 - Nov, 2023):}\par
    % Led the development of a Rollerball Bot for an AI course project in C++ for different sized boards with different type of pieces in which we integrated alpha-beta pruning and Monte Carlo Tree Search for state analysis and decision making.\par
    % Successfully competed in a class tournament, showcasing strong problem-solving and AI implementation skills where the bot chooses the best move to make within overall time and score constraints.

    \entry*
\textbf{AI-Powered Rollerball Bot (Prof. Mausam) (Sept, 2023 - Nov, 2023):}\par
Engineered a sophisticated Rollerball Bot in C++, leveraging alpha-beta pruning and Monte Carlo Tree Search algorithms to make optimal decisions across various board sizes and piece configurations.\par
Secured top ranks in a class tournament, underscoring strong problem-solving and AI implementation skills critical for real-world applications.\par
\textbf{GitHub Repository:} \href{https://github.com/nikhiltej5/player_AI.git}{\texttt{click here}}


    % \entry*
    % \textbf{ ERP database design| Course Project - IIT Delhi (Prof. Kaustubh Beedkar) (Jan, 2024 - feb, 2024):}\par
    %     Designed a database for course administration purposes after going through the requirements, ER modeling, and discussions with the Institute management.\par
    %     My first task as a database designer was to create an initial schema for all the tables with attributes and their abovementioned types.The requirement analysis also revealed certain behaviors and integrity the database must uphold when deployed. All modifications are integrated and interdependent.

    \entry*
\textbf{ERP Database Design and Implementation (Prof. Kaustubh Beedkar) (Jan, 2024 - Feb, 2024):}\par
    Orchestrated the design and deployment of a comprehensive database solution for course administration, collaborating closely with Institute management to ensure seamless integration and adherence to essential integrity constraints.\par
    Demonstrated expertise in ER modeling and database management, driving efficiencies and enhancing data-driven decision-making processes within the institution.\par
    \textbf{GitHub Repository:} \href{https://github.com/nikhiltej5/postgresql_erp.git}{\texttt{click here}}

    \entry*
\textbf{Innovative E-Learning Website Development ( Prof. Abhilash Jindal) (March, 2023 - April, 2023):}\par
    Spearheaded the development of Learnify, a cutting-edge e-learning platform, featuring seamless integration of frontend technologies with backend Python Flask framework and MySQL databases sourced from YouTube and Kaggle.\par
    Designed and implemented diverse REST APIs for enhanced user experience, including innovative features such as Voice Search, Translate ML API integration, and streamlined instructor onboarding processes.\par
    Elevated user engagement and accessibility through intuitive UI design and continuous frontend refinement, driving increased platform adoption and user satisfaction metrics.\par
    \textbf{GitHub Repository:} \href{https://github.com/nikhiltej5/Learnify.git}{\texttt{click here}}

%     \entry*
%     \textbf{Pipelining and Memory Hierarchy | Course Project- IIT Delhi (Prof. Riju Rekha Sen) (March, 2023 - April, 2023):}\par
%     Have programmed and complied code using C++ which runs to find time it takes to compile MIPS instructions on Hardware.\par
%     In order to speed up implementation of MIPS instructions Pipelining was used, Pipelining is carried out as 5-stage (IF,
% ID, EX, MEM, WB) with 4-registers (IF-ID, ID-EX, EX-MEM, MEM-WB). Data dependencies were worked by Stallings.\par
%     Write-back method was used to increase the Hit Rate in the Memory Hierarchy for read/write instructions. Memory
% Hierarchy is implemented as 3-stage (L1, L2, Mem) and each cache is n-way set associative with LRU rule.

    \entry*
    \textbf{Optimized Pipelining and Memory Hierarchy (Prof. Riju Rekha Sen) (March, 2023 - April, 2023):}\par
    Engineered a high-performance C++ instruction processor using advanced pipelining techniques and optimized memory hierarchy, resulting in significant reduction in compilation time and increased cache hit rates.\par
    Employed stallings and write-back strategies to mitigate data dependencies and enhance overall system throughput, showcasing proficiency in hardware-level optimizations and architectural design principles.

%     \entry*
%     \textbf{E-Learning website development | Course Project- IIT Delhi ( Prof. Abhilash Jindal) (March, 2023 - April, 2023):}\par
%         Developed Learnify, e-learning website and hosted it on the Baadal VM. Frontend is done using HTML, CSS,
% JavaScript and Backened with Python's Flask and data bases from YouTube and Kaggle worked with pyMySql\par
% Designed and included various rest API's such as Login/Signup, course by Category, Voice search/ Text search, Add
% Video/Review/Reply/Query, OTP, change password and included Translate ML API which is taken from a website\par
%     Users can become Instructors by going through a specifc process. UI design and frontend continuity is maintained.
\end{rubric}